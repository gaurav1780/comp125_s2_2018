\usepackage[T1]{fontenc}
\usepackage{pslatex}
 \usepackage[pdftex]{color}  
 \usepackage[pdftex]{graphicx}     
\usepackage{verbatim}
\usepackage{xcolor}
\usepackage{paralist}
\usepackage{tagging}

\usepackage[colorlinks=true,urlcolor=red]{hyperref}
\setlength{\topmargin}{-0.5in}                  % topmargin now at 1in
\setlength{\textheight}{9.5in}                  % body of text = 9.5in
\setlength{\oddsidemargin}{0in}                 % left margin = 1.0in on odd-numbered pages
\setlength{\evensidemargin}{0in}                % left margin = 1.0in on even-numbered pages 
\setlength{\textwidth}{6.5in}                   % width of text line.
\setlength{\parindent}{0.0in}
\newcommand{\code}{\texttt}

\usepackage{listings}
\lstset{%
	language=Java,
	basicstyle=\footnotesize\ttfamily,
	numbers=left,
	numberstyle=\tiny,        
	xleftmargin=17pt,
        	xrightmargin=5pt,
	frame=single,
	breaklines=true,
	moredelim=\item \item [is][\color{red}]{@}{@}
}

\begin{document}

\definecolor{aquamarine}{rgb}{0,0,0.7}
\definecolor{blue}{rgb}{0,0,0.7}
\definecolor{red}{rgb}{1,0,0}

%
\vspace{0.2in}
\begin{center}
        {\large  %MACQUARIE UNIVERSITY\\
%\medskip
\includegraphics[scale=0.3]{../../logo.png}\\
\medskip
        {\it  Faculty of Science and Engineering\\}
        \vspace{0.2in}
         {\bf COMP125 Fundamentals of Computer Science\\
        Workshop Week 9\\}}
\end{center}
\vspace{0.3in}
%

%\renewcommand{\labelenumi}{\arabic{enumi}.}
\renewcommand{\labelenumi}{\alph{enumi}.}
 
\section* {Learning outcomes}

By the end of this session, you will have learnt about recursions. 

%\begin{center}
%\fbox{
%\begin{minipage}{0.8\textwidth}
%Import project from \texttt{recursionWorkshopTemplate.zip}. 
%\end{minipage}
%}
%\end{center}

\begin{questions}

\vskip 0.5 cm 

\question  \textbf{Recursion trace} \vskip 0.5cm

Consider the following recursive function definition,

\begin{lstlisting}
int foo(int a) {
	if(a == 2)
		return 2;
	return a + foo(a / 2);
}
\end{lstlisting}

What is the value of variable \texttt{result} if the function call is,

\begin{lstlisting}
	int result = foo(16);
\end{lstlisting}

\begin{solution}
\begin{verbatim}
foo(16) = 16 + foo(8)
foo(8) = 8 + foo(4)
foo(4) = 4 + foo(2)
foo(2) = 2
Therefore,
foo(4) = 4 + 2 = 6
foo(8) = 8 + 6 = 14
foo(16) = 16 + 14 = 30
\end{verbatim}
\end{solution}

\vskip 0.5 cm 

\question  \textbf{Debugging recursive functions} \vskip 0.5cm

The following function attempts to compute the factorial of integer $n$. What is wrong with the function?

\begin{lstlisting}
int factorial(int n) {
	return n * factorial(n - 1);
	if(n == 0)
		return 0;
}
\end{lstlisting}

\begin{solution}
\begin{enumerate}
\item Termination conditions should be BEFORE recursive call.
\item Termination condition is incorrect. Should return 1 if $n \leq 1$ otherwise factorial will become 0.
\end{enumerate}

\begin{lstlisting}
int factorial(int n) {
	if(n <= 1)
		return 1;
	return n * factorial(n - 1);
}
\end{lstlisting}
	
\end{solution}

\vskip 0.5 cm 

\question  \textbf{Debugging recursive functions} \vskip 0.5cm

Give an example of a value, that, if passed to the function \texttt{foo} from the previous question, calls itself indefinitely.

\begin{solution}
foo(24) = 24 + foo(12)
foo(12) = 12 + foo(6)
foo(6) = 6 + foo(3)
foo(3) = 3 + foo(1)
foo(1) = 1 + foo(0)
foo(0) = 0 + foo(0)
...	
\end{solution}

\vskip 0.5 cm

\question  \textbf{Some more recursive trace} \vskip 0.5cm

Consider the following recursive function definition,

\begin{lstlisting}
int foo(int a) {
	if(a <= 0)
		return 0;
	if(a % 2 == 0)
		return foo(a/2);
	else
		return 1 + foo(a/2);
}
\end{lstlisting}

What is the value of variable \texttt{result} if the function call is,

\begin{lstlisting}
	int result = foo(59);
\end{lstlisting}

\begin{solution}
\begin{verbatim}
foo(59) = 1 + foo(29)
foo(29) = 1 + foo(14)
foo(14) = foo(7)
foo(7) = 1 + foo(3)
foo(3) = 1 + foo(1)
foo(1) = 1 + foo(0)
foo(0) = 0
Therefore, foo(59) = 5
\end{verbatim}
\end{solution}

\vskip 0.5 cm

\question  \textbf{Writing a recursive function} \vskip 0.5cm
Write a recursive function, that when passed an integer, returns the number of even digits in that integer. Return 0 if the integer is 0.

\begin{solution}
\begin{lstlisting}
int nEvenDigits(int n) {
	if(n == 0)
		return 0;
	if(n%2 == 0)
		return 1 + nEvenDigits(n/10);
	else
		return nEvenDigits(n/10);
}
\end{lstlisting}	
\end{solution}

\vskip 0.5 cm

\question  \textbf{Writing a recursive function} \vskip 0.5cm
Write a recursive function, that when passed an integer $n$, return the sum of squares of the first $n$ positive integers  ($1 + 2 + ... + n$).

\begin{solution}
\begin{lstlisting}
int sumSquares(int n) {
	if(n <= 0)
		return 0;
	return n*n + sumSquares(n - 1);
}
\end{lstlisting}	
\end{solution}

\vskip 0.5 cm

\question  \textbf{Writing a recursive function dealing with text} \vskip 0.5cm
Write a recursive function, that when passed a String, returns the number of digits in the String.

\begin{solution}
\begin{lstlisting}
int nDigits(String s) {
	if(s == null || s.length() == 0)
		return 0;
	if(s.charAt(0) >= '0' && s.charAt(0) <= '9')
		return 1 + nDigits(s.substring(1));
		return nDigits(s.substring(1));	
}
\end{lstlisting}	
\end{solution}

\question  \textbf{Counting recursive function calls} \vskip 0.5cm

How many calls are made to \texttt{gcd} if the original call is \texttt{gcd(30, 72}?

\begin{lstlisting}
int gcd(int a, int b) {
	if(a < b) 
		return gcd(b, a);
	if(b == 0)
		return a;
	return gcd(b, a%b);
\end{lstlisting}

\begin{solution}
\begin{verbatim}
gcd(30, 72) = gd(72, 30)
gcd(72, 30) = gcd(30, 12)
gcd(30, 12) = gcd(12, 6)
gcd(12, 6) = gcd(6, 0)
gcd(6, 0) = 6

a total of 5 function calls
\end{verbatim}
\end{solution}

\vskip 0.5 cm \question  \textbf{(Tracing slightly more complex recursive functions)} \vskip 0.5cm
Consider the definition of the following recursive function,

\begin{lstlisting}
public static void displayBrackets(int n) {
	if(n == 0)
		return;
	System.out.print("{");
	for(int i=0; i < n - 2; i++) {
		displayBrackets(n - 1);
		System.out.print(", ");
	}
	displayBrackets(n - 1);
	System.out.print("}");	
}
\end{lstlisting}

What is the output of the following statement?

\begin{lstlisting}	
displayBrackets(3);
\end{lstlisting}
	
\vskip 0.5 cm

\question  \textbf{(Assessed task) Defining recursive functions} \vskip 0.5cm

I have made up a sequence called a \emph{tribonacci} sequence. 
The first three numbers of this sequence are 1, 2 and 3, and every subsequent number in this sequence is the sum of the previous \textbf{three} numbers. Thus, the sequence is $1, 2, 3, 6, 11, 20, 37, 68, ...$. Write a function to compute the $n^{th}$ \emph{tribonacci} number. Assuming the $1^{st}$ number is 1.

\begin{solution}
\begin{lstlisting}
int tribonacci(int n) {
	if(n < 4)
		return n;
	return tribonacci(n - 1) + tribonacci(n - 2) + tribonacci(n - 3);
}
\end{lstlisting}	
\end{solution}

\question  \textbf{(Assessed task) Counting recursive function calls} \vskip 0.5cm

How many calls are made to \texttt{tribonacci} if the original call is \texttt{tribonacci(5)}?

\begin{solution}
\begin{verbatim}
t(5) = t(4) + t(3) + t(2)
t(4) = t(3) + t(2) + (t1)
t(3) called twice returns 3 each time
t(2) called twice returns 2 each time
t(1) called once returns 1
So termination level calls are 5, and in addition t(4) and t(5) once each. So a total of 7 function calls.
\end{verbatim}
\end{solution}

\question  \textbf{(Voluntary Assessed task) Writing a recursive function} \vskip 0.5cm

Write a recursive function that displays an hour-glass pattern. For example, it displays the following pattern for $n = 5$.

\begin{verbatim}
***********
 *********
  *******
   *****
    ***
     *
     *
    ***
   *****
  *******
 *********
***********
\end{verbatim}

And it displays the following pattern for $n = 7$.

\begin{verbatim}
***************
 *************
  ***********
   *********
    *******
     *****	
      ***
       *
       *
      ***
     *****
    *******
   *********
  ***********
 *************
***************
\end{verbatim}



\end{questions}
\end{document}
