\input{comp125workshopHeader}
\section* {Learning outcomes}

By the end of this session, you will have learnt the basics about containers and \texttt{ArrayList} class. 

\section*{Questions}
\begin{questions}

\question Why do we need classes that hold collections when we already have arrays? Give three reasons.

\ifprintanswers
\begin{enumerate}
\item We have to re-size arrays manually if they get full and more items need to be added.
\item Everything on arrays needs to be done using just \texttt{arr.length} and \texttt{arr[i]} operations.
\item We might want to customize what operations we want over our collection, and their interpretation.
\item We might want a more specific ordering in our collection.
\end{enumerate}
\else
\fi

\question
An \texttt{ArrayList} is a resizable collection of objects. If you don't parameterise an \texttt{ArrayList}, it can hold a variety of objects. That is, each item of the \texttt{ArrayList} can be of a different class.

A parameter-less \texttt{ArrayList} is created as -

\begin{lstlisting}
    ArrayList list = new ArrayList();
\end{lstlisting}
where \texttt{list} is the \texttt{ArrayList} object.

You can parameterize an \texttt{ArrayList} so that it stores objects of a specific class. A parameterized \texttt{ArrayList} is created as -

\begin{lstlisting}
    ArrayList<ClassType> list = new ArrayList();
\end{lstlisting}

where \texttt{list} is the \texttt{ArrayList} object.

\newpage

For example,

\begin{lstlisting}
    ArrayList<String> list = new ArrayList();
\end{lstlisting}

can only hold \texttt{String} objects.

A subset of methods (the important ones) applicable to an \texttt{ArrayList} object is given below -

\begin{itemize}
\item \texttt{int size()}: returns the number of items in the list
\item \texttt{Object get(int index)}: returns the \texttt{Object} at the specified index, if any; and \texttt{null} otherwise.
\item \texttt{add(Object obj)}: adds the specified \texttt{Object} to the end of the list and returns \texttt{true}, if it can; and \texttt{false} otherwise.
\item \texttt{add(int idx, Object obj)}: adds the specified \texttt{Object} at given index. Shifts all items at index \texttt{idx} onwards to the right.
\item \texttt{contains(Object obj)}: returns \texttt{true} if the specified exists, and \texttt{false} otherwise.
\item \texttt{indexOf(Object obj)}: returns the index of the specified \texttt{Object} if it exists, and -1 otherwise.
\item \texttt{remove(Object obj)}: removes the specified \texttt{Object} to the list and returns \texttt{true}, if it can; and \texttt{false} otherwise.
\item \texttt{set(int index, Object obj)}: updates the item at given index to the object passed. Returns the item that the new object has replaced.
\end{itemize}

Write a piece of code that performs the following operations in the given order -

\begin{enumerate}
	\item Create an \texttt{ArrayList list} to hold \texttt{String} objects
	\item Add "hello" to \texttt{list}
	\item Add "this" to \texttt{list}
	\item Add "is" to \texttt{list}
	\item Add "your" to \texttt{list}
	\item Add "captain" to \texttt{list}
	\item Add "speaking" to \texttt{list}
	\item Remove the 5th item (at index 4) from \texttt{list}
	\item Insert "brother" at index 4 in \texttt{list}.
	\item Change the 6th item (at index 5) to "talking"
	\item Display the number of items in \texttt{list}
	\item Display all items of the \texttt{list}
	\item Display each item in \texttt{list} on a separate line.
	\item Store in a variable \texttt{loc} the index where ``brother'' is found in the \texttt{list}, and display it.
	\item Display the first character of each item of the list
	\item Create a String consisting of the first characters of each item. For example, if the items are ``this'', ``is'', ``fun'', your String should be ``tif''
	\item Count the number of items that begin with an `t' or `T'
	\item Count the number of items that are more than 3 characters long
	\item Create an arraylist of items that are more than 3 characters long and display it
	\item Create a \texttt{char} array consisting of the last characters of each item. For example, if the items are ``this'', ``is'', ``fun'', your array should be $\{`s', `s', `n'\}$
	\item Replace each item by their uppercase version, that is capitalize all Strings
\end{enumerate}

\begin{solution}
\begin{lstlisting}
ArrayList<String> list = new ArrayList();
list.add("hello");
list.add("this");
list.add("is");
list.add("your");
list.add("captain");
list.add("speaking");
list.remove(4);
list.add(4, brother);
list.set(5, "talking");
System.out.println(list.size());
System.out.println(list);
for(int i=0; i<list.size(); i++)
    System.out.println(list.get(i));
int loc = list.indexOf("done");
for(int i=0; i<list.size(); i++)
	System.out.println("item "+(i+1)+": "+list.get(i));
System.out.println();

for(int i=0; i<list.size(); i++)
	System.out.println("First character of item "+(i+1)+": "+list.get(i).charAt(0));
System.out.println();

int count = 0;
for(int i=0; i<list.size(); i++)
	if(list.get(i).substring(0, 1).equalsIgnoreCase("a"))
		count++;
System.out.println(count+" items begin with 'a'");

count = 0;
for(int i=0; i<list.size(); i++)
	if(list.get(i).length() > 3)
		count++;
System.out.print(count+" items are longer than 3 characters: ");

LinkedList<String> longOnes = new LinkedList<String>();
for(int i=0; i<list.size(); i++)
	if(list.get(i).length() > 3)
		longOnes.add(list.get(i));
System.out.println(longOnes);

String firstChars = "";
for(int i=0; i<list.size(); i++)
	firstChars+=list.get(i).charAt(0);
System.out.println("String of first characters: "+firstChars);

String lastChars = "";
for(int i=0; i<list.size(); i++)
	lastChars+=list.get(i).charAt(list.get(i).length()-1);
System.out.println("String of last characters: "+lastChars);

for(int i=0; i<list.size(); i++)
	list.set(i, list.get(i).toUpperCase());
System.out.println("Capitalized list: "+list);
\end{lstlisting}
\end{solution}

\question Following is an incomplete class definition for a custom-built array-based list. Complete the method \texttt{removeFirst}. Details provided as javadoc method header comment.

\ifprintanswers
\begin{lstlisting}
public class MyArrayList {
	private double[] data;
	private int nItems;
	...
	/**
	remove first item in the list, if any and
	return it from the method
	
	return null if list is empty
	*/
	public Double removeFirst() {
		if(nItems == 0) 
			return null;
		double removedItem = data[0];
		for(int i=1; i < nItems; i++) {
			data[i-1] = data[i];
		}
		nItems--;
		return removedItem;
	}
}
\end{lstlisting}
\else
\begin{lstlisting}
public class MyArrayList {
	private double[] data;
	private int nItems;
	...
	/**
	remove first item in the list, if any and
	return it from the method
	
	return null if list is empty
	*/
	public Double removeFirst() {
		//to be completed
	}
}
\end{lstlisting}
\fi

\question Add a method \texttt{product} that when passed an \texttt{ArrayList} of \texttt{Double} objects, returns the product of all items in the \texttt{ArrayList}. The method should return 0 if the list is \texttt{null} or empty.

\begin{lstlisting}
public static double product(ArrayList <Double> list)
\end{lstlisting}

\begin{solution}
\begin{lstlisting}
public static double product(ArrayList <Double> list) {
	if(list == null || list.size() == 0) 
		return 0;
	double result = 1;
	for(Double item: list)
		result*=item;
	return result;
}	
\end{lstlisting}	
\end{solution}


\question Add a method \texttt{sumPositive} that when passed an \texttt{ArrayList} of \texttt{Integer} objects, returns the sum of all positive values in the \texttt{ArrayList}. The method should return 0 if the list is \texttt{null} or empty.

\begin{lstlisting}
public static int sumPositive(ArrayList <Integer> list)
\end{lstlisting}

\begin{solution}
\begin{lstlisting}
public static int sumPositive(ArrayList <Integer> list) {
	if(list == null || list.size() == 0) 
		return 0;
	int result = 0;
	for(Integer item: list)
		if(item > 0)			
			result+=item;
	return result;
}	
\end{lstlisting}	
\end{solution}

\question Add a method \texttt{count} that when passed an \texttt{ArrayList<Integer> list} and an \texttt{Integer target}, returns the number of times \texttt{target} exists in \texttt{list}.

\begin{lstlisting}
public static int count(ArrayList<Integer> list, Integer target)
\end{lstlisting}

\begin{solution}
\begin{lstlisting}
int count(ArrayList<Integer> list, Integer target) {
	if(list == null)
		return 0;
	int result = 0;
	for(Integer item: list)
		if(item == target)
			result++;
	return result;
}
\end{lstlisting}
\end{solution}

\question Write a method that when passed an \texttt{ArrayList} of characters, returns an array containing the characters of the \texttt{ArrayList}. For example, if the \texttt{ArrayList passed} is \texttt{[`v', `e', `n', `d', `e', `t', `t', `a']}, the array returned should be \texttt{\{`v', `e', `n', `d', `e', `t', `t', `a'\}}. You may NOT use built-in methods to convert an \texttt{ArrayList} to an array.

\begin{solution}
\begin{lstlisting}
public static char[] toArray(ArrayList <Character> list) {
	if(list == null) 
		return null;
	char[] result = new char[list.size()];
		int i = 0;
		for(Character item: list) {
			result[i] = item;
			i++;
		}
	return result;
}	
\end{lstlisting}	
\end{solution}


\question Complete the method \texttt{squared} that when passed an \texttt{ArrayList<Integer> list}, squares all items of \texttt{list}. So if the list that is passed is \texttt{[3, 1, 7]}, after the method executes, it becomes \texttt{[9, 1, 49]}.

Hint 1: the method on \texttt{ArrayList} that you'll need are,

\begin{itemize}
\item \texttt{size()}
\item\texttt{get(int index)}
\item \texttt{set(int index, int value)}
\end{itemize}

\begin{lstlisting}
public static void squared(ArrayList<Integer> list)
\end{lstlisting}

\begin{solution}
\begin{lstlisting}
void squared(ArrayList<Integer> list) {
	if(list == null)
		return;

	for(int i=0; i<list.size(); i++)
		list.set(i, list.get(i) * list.get(i));
}
\end{lstlisting}
\end{solution}

\question \textbf{(Challenging)} Write a method that when passed an arraylist of arraylists of integers, returns an arraylist containing items that are exclusive to each list. For example, if the list passed is [[8, 1, 4, 2, 4, 2, 1], [6, 4, 9, 8, 8, 8], [5, 3, 8, 8, 5, 6]], the method should return an \texttt{ArrayList} containing [1, 2, 2, 1, 9, 5, 3, 5]

\begin{solution}
\begin{lstlisting}
public static ArrayList<Integer> exclusiveItems(ArrayList<ArrayList<Integer>> megaList) {
	ArrayList<Integer> result = new ArrayList();
	for(int i=0; i < megaList.size(); i++) {
		for(Integer item: megaList.get(i)) {
			boolean dup = false;
			for(int k=0; k < megaList.size() && !dup; k++) {
				if(i != k && megaList.get(k).contains(item)) {
					dup = true;
				}	
			}
			if(!dup) 
				result.add(item);
		}
	}
	return result;
}
\end{lstlisting}
\end{solution}

\question Discuss the time complexities of following operations on an arraylist (best and worst cases).

\begin{enumerate}
  \item inserting item at an arbitrary position
  \item removing item from an arbitrary position
  \item accessing item at an arbitrary position
\end{enumerate}

\ifprintanswers
\Large
\begin{center}
  \begin{tabular}{c|c|c}
    Operation & Best case & Worst case\\
    \hline
    Accessing an item &  $\mathcal{O}(1)$ & \color{green!90!black} $\mathcal{O}(1)$ \color{black}\\
    \hline
    Inserting an item & \color{green!90!black} $\mathcal{O}(1)$ \color{black} & \color{red} $\mathcal{O}(n)$ \color{black} \\
    \hline
    Removing an item & \color{green!90!black} $\mathcal{O}(1)$ \color{black} & \color{red} $\mathcal{O}(n)$ \color{black}
  \end{tabular}
\end{center}
\normalsize
\else
\fi
\end{questions}
	
\end{document}
